\documentclass[a4paper]{article}

\usepackage[utf8]{inputenc}
\usepackage[portuges]{babel}
\usepackage{graphicx}
\usepackage{a4wide}
\usepackage[pdftex]{hyperref}
\usepackage{float}
\usepackage{graphicx}
\usepackage{indentfirst}


\begin{document}

\title{Projeto de Laboratórios de Informática III\\Grupo 1}
\author{Catarina Machado (a81047) \and Cecília Soares (a34900) \and João Vilaça (a82339)}
\date{\today}

\begin{titlepage}

  %título
  \thispagestyle{empty}
  \begin{center}
  \begin{minipage}{0.75\linewidth}
      \centering
  %engenharia logo
      \includegraphics[width=0.4\textwidth]{eng.jpeg}\par\vspace{1cm}
      \vspace{1.5cm}
  %titulos
      \href{https://www.uminho.pt/PT}{\scshape\LARGE Universidade do Minho} \par
      \vspace{1cm}
      \href{https://www.di.uminho.pt/}{\scshape\Large Departamento de Informática} \par
      \vspace{1.5cm}

  \maketitle

  \end{minipage}
  \end{center}
  \clearpage

 \end{titlepage}


\begin{abstract}
O presente relatório descreve o projeto realizado no âmbito da disciplina de
\href{http://miei.di.uminho.pt/plano_estudos.html#laborat_rios_de_inform_tica_iii}
{\emph {Laboratórios de Informática III} (LI3)}, ao longo do segundo semestre,
do segundo ano, do \href{http://miei.di.uminho.pt}{Mestrado Integrado em Engenharia Informática}
da \href{https://www.uminho.pt}{Universidade do Minho}.

O objetivo do projeto foi desenvolver um sistema capaz de processar a informação
contida em ficheiros XML para responder a um conjunto de interrogações de forma
eficiente, utilizando, para isso, a linguagem de programação C. Neste documento
descrevemos sucintamente as decisões de implementação e as limitações da solução
adoptada.

\end{abstract}
\pagebreak

\tableofcontents


\pagebreak

\section{Introdução}
\label{sec:intro}

O presente relatório foi elaborado no âmbito da unidade curricular
\href{http://miei.di.uminho.pt/plano_estudos.html#laborat_rios_de_inform_tica_iii}
{\emph {Laboratórios de Informática III} (LI3)}, ao longo do segundo semestre,
do segundo ano, do \href{http://miei.di.uminho.pt}{Mestrado Integrado em Engenharia Informática}
da \href{https://www.uminho.pt}{Universidade do Minho}, e tem como objetivo
descrever as tarefas desenvolvidas para criar um sistema capaz de processar as
informações contidas em ficheiros XML para responder a um conjunto de
interrogações de forma eficiente, utilizando a linguagem de programação C.


\subsection{Descrição do Problema}
\label{sec:problema}

Os ficheiros em formato XML que deveriam ser processados continham informação referente ao
website \href{https://stackoverflow.com/}{\textit{Stack Overflow}}, o qual foi
criado em 2008 por Jeff Atwood e Joel Spolsky para que programadores e
entusiastas possam expor as suas dúvidas, funcionando como um fórum em que os usuários fazem
perguntas e respondem a dúvidas dos seus pares. Além disso, podem ainda votar
em questões e respostas que considerem pertinentes, obtendo pontos de reputação
e medalhas pela qualidade da sua intervenção.

Em concreto, o trabalho prendia-se em extrair a informação necessária dos vários
ficheiros, de forma a conseguirmos responder a 11 interrogações relacionadas com
o conteúdo dos mesmos da forma mais eficiente possível, isto é, tendo especial
atenção ao tempo de execução do programa, bem como ao encapsulamento dos dados.

\subsection{Ficheiros XML}
\label{sec:xml}

De todos os ficheiros XML colocados à nossa disposição (Votes, Tags, Users,
PostLinks, Posts, PostHistory, Comments e Badges) e que se destinavam a dar resposta
a um conjunto de interrogações, decidimos que apenas iríamos precisar de carregar
algumas informações contidas nos ficheiros Tags.xml, Users.xml e Posts.xml. \par
No ficheiro Tags.xml retiramos somente o identificar da tag (Id) e o nome da mesma
(TagName). No ficheiro Users.xml, recolhemos a informação relativa ao identificador
do utilizador (Id), à sua reputação (Reputation), ao seu nome (DisplayName) e ao seu
perfil (AboutMe). \par
Por último, quanto ao ficheiro Posts.xml extraímos o identificador
do post (Id); o tipo de post (PostTypeId); o utilizador que publicou o post
(OwnerUserId); o título do post (Title); as suas tags (Tags); a pontuação obtida
(Score); o número de comentários que foram feitos (CommentCount); no caso de
ser uma resposta, o número da pergunta a que se refere (ParentId) e, finalmente,
a sua data de criação (CreationDate).

\subsection{Concepção da Solução}
\label{sec:solucao}

Para resolvermos este problema foram cruciais três momentos. Numa primeira fase,
definimos a estrutura de dados que consideramos que melhor solucionaria o nosso
problema. Posteriormente, analisamos as vantagens e desvantagens das diferentes
alternativas para fazer a leitura dos dados e a recolha da informação relevante.
A nossa opção recaiu na utilização da API SAX para fazer o \textit{parser} da
informação, dado que permite o acesso serial ao conteúdo de um documento XML de
forma orientada a eventos, aquilo a que chamam \textit{event-based parser} para
os documentos XML.
Finalmente, na última etapa do projeto concentramo-nos em responder
às diferentes \textit{queries}.

O código subjacente à solução proposta pode ser encontrado no repositório:

\begin{center}
\href{https://github.com/dium-li3/Grupo1}{\emph{https://github.com/dium-li3/Grupo1}}.
\end{center}

O restante deste relatório está organizado da seguinte forma: a
Secção~\ref{sec:estruturadedados} descreve as estruturas de dados adoptadas,
ao passo que a Secção~\ref{sec:implementacao}  apresenta e discute a solução
proposta para a resolução do problema. O relatório termina com conclusões na
Secção~\ref{sec:conclusao}, onde é também apresentada uma análise crítica dos
resultados obtidos.



\section{Organização dos Dados}
\label{sec:estruturadedados}

Para desenvolvermos este trabalho adotamos as seguintes estruturas de dados:

\subsection{Tipo de Dados Concretos}
\label{sec:dados_concretos}

\begin{verbatim}
typedef struct TCD_community {
  GHashTable * users;
  GHashTable * questions;
  GHashTable * answers;
  GList * questionsList;
  GList * usersList;
  GPtrArray * day;
  GHashTable * tags;
} TCD_community;
\end{verbatim}

\vspace{0.2cm}

Os dados necessários para responder às queries foram armazenados
na estrutura \texttt{TCD\_community}, a estrutura de dados principal do nosso trabalho.
Esta estrutura armazena outras 7 estruturas que serão explicitadas mais tarde.
De modo a respondermos às queries da forma mais eficiente possível
utilizamos estruturas de dados já existentes e, para isso, recorremos ao glib,
uma biblioteca que fornece simples estruturas de dados para programas em C. \par

Em primeiro lugar, para as estruturas \textbf{users}, \textbf{questions},
\textbf{answers} e \textbf{tags}
utilizamos uma GHashTable para armazenar os dados. \par
A razão para essa escolha deve-se ao facto de tantos os ids dos users, como os
ids das questions, answers e tags não estarem armazenados nos ficheiros de forma
sequencial, nem serem contíguos (existem ``buracos'' entre os respetivos números
identificadores), pelo que se utilizássemos os ids como sendo os índices de um
array haveria muita memória desperdiçada, e, por isso, não vimos nenhuma vantagem
em utilizar um GPtrArray ou uma lista, nem mesmo utilizando algoritmos auxiliares,
como por exemplo o algoritmo de procura binária.

No caso das árvores binárias, o tempo de procura e inserção no caso médio é
logarítmico e no pior caso linear, o que poderia ser uma opção viável, porém,
concluímos que utilizando GHashTable, ou seja, hash tables, e recorrendo aos ids
dos users, das questions e das answers, e ao nome da tag no caso das tags, como
meio de procura dos elementos (chave da hash) conseguimos um tempo médio de
inserção e procura (os dois métodos que mais utilizamos) constante, e no pior
dos casos conseguimos um tempo linear ao tamanho do array, o que se traduz numa
forma muito mais rápida e simples de trabalharmos o problema pois conseguimos
aceder aos elementos da estrutura de dados muito eficientemente com o
recurso a funções já existentes no glib. \par

No entanto, no caso das perguntas é também utilizada um segunda estrutura,
GList, criada aquando da primeira vez que é preciso utilizar perguntas
ordenadas por cronologia. A utilização desta estrutura permite que a inserção
de perguntas seja feita numa GHashTable, com complexidade média O(1),
e, apenas no fim, sejam ordenadas por data na GList, reduzindo a complexidade.\par

Utilizamos também uma segunda estrutura GList para os utilizadores, igualmente
criada aquanda da primeira vez que é preciso utilizar os utilizadores ordenados por 
reputação.\par

Na estrutura \textbf{day} utilizamos um GPtrArray porque consideramos que seria
a forma mais eficiente de percorrer os intervalos de tempo. Deste modo,
a estratégia que utilizamos foi criar um array de apontadores onde o
índice 0 corresponde à data de criação do stackoverflow (15/9/2008),
o índice 1 ao dia seguinte, e assim sucessivamente. Assim, para percorrermos
um intervalo de tempo basta fazermos 2 cálculos: 1- determinar qual é o 
índice da data inicial do intervalo no array (e para isso utilizamos uma função do glib
(g\_date\_days\_between) que nos diz quantos dias passaram entre 2 datas;
neste caso, sabendo quantos dias passaram desde o início do stackoverflow
até à data pretendida
sabemos qual é o índice dessa data). 2- Descobrirmos também qual é o número de
dias do intervalo de tempo. Depois disso, vamos ao GPtrArray, ao índice da
data inicial, e vamos percorrendo as restantes datas até à data final
bastando para isso incrementar 1 ao índice inicial até fazermos
um número de iterações igual ao número de dias do intervalo de tempo.
Com isto, conseguimos aceder às informações de qualquer dia e, consequentemente,
de qualquer intervalo de tempo de uma forma muito rápida e direta,
com um tempo de procura constante (O(1)). \par


\subsection{Estruturas de dados Complementares}
\label{sec:dados_complementares}

\begin{verbatim}
typedef struct users {
  long user_id;
  char * shortbio;
  char * username;
  int reputation;
  int n_posts;
  GArray * last_posts;
} users;
\end{verbatim}

A estrutura de dados \texttt{users} contém o id do utilizador, a sua short bio, o seu nome,
a sua reputação, o número total de posts desse utilizador (i.e. perguntas e respostas),
bem como um GArray contendo todos os posts do utilizador (perguntas e respostas) no
formato \textit{postAndDate}.
Esta estrutura encerra informações necessárias para responder às interrogações 1,
5, 8 e 10.

\begin{verbatim}
typedef struct questions {
  long post_id;
  postDate pd;
  long user_id;
  char * title;
  char * tags;
  int n_answers;
  int n_answers_date;
  int n_answer_votes;
  GPtrArray * answers;
} questions;
\end{verbatim}

A estrutura de dados \texttt{questions} contém o id da pergunta, a sua data no formato
\textit{postAndDate}, o id do utilizador que publicou a pergunta, o seu título,
as suas tags, o número total de
respostas que obteve, uma variável que servirá para armazenar
o número de respostas que a pergunta obtive durante um intervalo de tempo,
o número total de votos das suas respostas,
bem como um GPtrArray com todas as respostas de cada
pergunta no formato formato \textit{answers}.
Esta estrutura encerra informações necessárias para responder às interrogações 1,
4, 7, 8 e 11.

\begin{verbatim}
typedef struct answers {
  long user_id;
  long answer_id;
  long parent_id;
  int score;
  int comment_count;
} answers;
\end{verbatim}

A estrutura de dados \texttt{answers} contém o id do utilizador, o id da resposta, o id da
pergunta a que essa resposta se refere, a sua pontuação e os seus comentários.
Esta estrutura encerra informações necessárias para responder às interrogações 1,
6, 7, 9 e 10.

\begin{verbatim}
typedef struct day {
  int day;
  int month;
  int year;
  int n_questions;
  int n_answers;
  GPtrArray * questions;
  GPtrArray * answers;
} day;
\end{verbatim}

A estrutura de dados \texttt{day} contém uma data e as informações relativas à mesma,
nomeadamente o seu dia, mês, ano, o número de respostas e o
número de perguntas efetuadas nesse dia, um GPtrArray com todas as perguntas
desse dia no formato \textit{questions} e um GPtrArray com todas as respostas
desse dia no formato \textit{answers}. Esta estrutura encerra informações
necessárias para responder às interrogações 3, 4, 6, 7 e 11
(interrogações que dizem respeito a intervalos arbitrários de tempo).

\begin{verbatim}
typedef struct postAndDate {
    long post_id;
    int year, month, day, hour, min, sec, mili;
} postAndDate;
\end{verbatim}

A estrutura de dados \texttt{postAndDate} contém o id do post e o ano, mês, dia,
hora, minuto, segundo e milissegundo em que esse post
(pergunta ou resposta) foi efetuado. Esta estrutura encerra informações
necessárias para responder à interrogação 7.

\begin{verbatim}
typedef struct tags {
  int id;
  char * nameTag;
  int value;
} tags;
\end{verbatim}

A estrutura de dados \texttt{tags} contém o id que identifica a tag,
o nome da tag e ainda uma variável que servirá para armazenar
o número de ocorrências dessa tag durante um intervalo de tempo.
Esta estrutura encerra informações necessárias para responder à interrogação 11.


\section{Implementação}
\label{sec:implementacao}

\subsection{Modularização Funcional}
\label{sec:organizacao}

O trabalho é composto por vários módulos, tendo sido os que a seguir se descrevem
criados por nós, consoante as necessidades que fomos detectando.

\begin{itemize}
\begin{item} main.c - main do programa.\end{item}
\begin{item} 00load.c - contém as funções necessárias para carregar e processar os
dados necessários dos ficheiros XML.\end{item}
\begin{item} answers.c - todas as funções necessárias para aceder à estrutura de
dados answers.\end{item}
\begin{item} day.c - todas as funções necessárias para aceder à estrutura de
dados day.\end{item}
\begin{item} questions.c - todas as funções necessárias para aceder à estrutura de
dados questions.\end{item}
\begin{item} struct.c - todas as funções necessárias para inicializar e aceder à
estrutura de dados TCD\_community.\end{item}
\begin{item} tags.c - todas as funções necessárias para aceder à estrutura de
dados tags.\end{item}
\begin{item} users.c - todas as funções necessárias para aceder à estrutura de
dados users.\end{item}
\begin{item} postDate.c - todas as funções necessárias para aceder à estrutura de
dados postDate.\end{item}
\begin{item} 01TitleUserName.c - responde à interrogação n.º 1.\end{item}
\begin{item} 02TopMostActive.c - responde à interrogação n.º 2.\end{item}
\begin{item} 03totalPostDate.c - responde à interrogação n.º 3.\end{item}
\begin{item} 04questionsWithTag.c - responde à interrogação n.º 4.\end{item}
\begin{item} 05UserInfo.c - responde à interrogação n.º 5.\end{item}
\begin{item} 06mostVotedAnswers.c - responde à interrogação n.º 6.\end{item}
\begin{item} 07mostAnsweredQuestions.c - responde à interrogação n.º 7.\end{item}
\begin{item} 08titlesWithWord.c - responde à interrogação n.º 8.\end{item}
\begin{item} 09bothParticipated.c - responde à interrogação n.º 9.\end{item}
\begin{item} 10BetterAnswer.c - responde à interrogação n.º 10.\end{item}
\begin{item} 11mostUsedBestRep.c - responde à interrogação n.º 11.\end{item}
\begin{item} 12clean.c - liberta o espaço de memória antes de encerrar o programa.\end{item}
\end{itemize}

Os módulos por nós criados podem ser vistos como unidades interdependentes que se
complementam, cada uma com objetivos específicos, que interagem e que estão ligadas
entre si apenas por funções (única interface), garantindo o encapsulamento dos dados.

Cada módulo tem uma única funcionalidade, um objetivo específico,
como, por exemplo, aceder a uma estrutura de dados ou responder a uma query.
De facto, com a divisão física dos ficheiros tentamos distribuir as
tarefas do nosso programa de modo a garantir a abstração dos dados, bem como
facilitar a reutilização do código.

Essa interdependência ou coesão entre os módulos é bem patente, já que
os diferentes módulos necessitam interatuar para serem executados. Por exemplo,
todas as interrogações necessitam do módulo struct.h, que, por sua vez, depende dos módulos
interface.h, users.h, questions.h, postDate.h, day.h e tags.h, os quais dependem de vários
outros e assim sucessivamente.

Contudo, apesar de haver uma interdependência entre os módulos, eles estão
fracamente ligados na medida em que a única ligação entre eles é através da interface de
cada módulo, isto é, recorrendo somente às diferentes funções disponibilizadas e
nunca acedendo diretamente às estruturas de dados criadas.


\subsection{Abstração de Dados}
\label{sec:abstracao}

Conforme descrevemos na secção anterior, uma das estratégias adotadas para garantir
a abstração dos dados foi através da modularização funcional do código.
Aliado a isso, o encapsulamento de todos os dados foi garantido por \textbf{\textit{forward declaration}}
de todas as estruturas definidas, sendo os seus atributos somente acessíveis através
de funções ``getters e setters''. De facto, os ficheiros headers criados impossibilitam
que os atributos das diversas estruturas de dados criadas sejam acedidos diretamente.
Por exemplo, a estrutura de dados que contém todos as informações relevantes e
referentes ao usuário foi definida no ficheiro users.c da seguinte forma:

\begin{verbatim}
typedef struct users {
  long user_id;
  char * shortbio;
  char * username;
  int reputation;
  int n_posts;
  GArray * last_posts;
} users;
\end{verbatim}

sendo que o acesso aos elementos de users só é possível através das funções
``getters e setters'', como por exemplo, \textit{getUserId} ou
\textit{setUserId}, já que o ficheiro header users.h possui apenas a declaração
\textit{typedef struct users * Users}.


\subsection{Queries}
\label{sec:queries}

Finalmente, a última etapa do nosso projeto prendeu-se com a resposta às diversas
\textit{queries}, as quais passamos a explicar neste capítulo.


\subsubsection*{Query 1}
\label{sec:query1}

\textbf{“Dado o identificador de um post, a função deve retornar
um par com o título do post e o nome (não o ID) de utilizador do autor. Se o post
for uma resposta, a função deverá retornar informações (título e utilizador)
da pergunta correspondente.”}xw

Na resposta a esta query, começamos por verificar se o id do post passado como
parâmetro  para a função identifica uma pergunta. No caso da resposta ser negativa
verificamos se esse id identifica alguma resposta contida 
na GHashTable \texttt{answers}. \par
Na hipótese de ser uma pergunta, com o auxílio da estrutura de dados \texttt{questions}
conseguimos obter o título do post e o id do utilizador, através do qual, procurando
na GHashTable users conseguimos saber se o usuário existe e, caso exista, obter
todas as suas informações relevantes e, posteriormente, extrair o seu nome através
da função getUsername. Desta forma, conseguimos devolver o par
pretendido, o título do post e o nome do utilizador. \par
No caso do id passado como parâmetro para a função identificar uma resposta, determinamos
a que pergunta esta se refere e seguimos o procedimento descrito no parágrafo anterior.

\subsubsection*{Query 2}
\label{sec:query2}

\textbf{“Função que devolve o top N utilizadores com maior número
de posts de sempre. Para isto, são considerados tanto perguntas
quanto respostas dadas pelo respectivo utilizador.”}

\begin{verbatim}
typedef struct totalPosts {
  long user_id;
  int n_posts;
} totalPosts;
\end{verbatim}

Para respondermos a esta interrogação criamos a estrutura de dados
\texttt{totalPosts}. Esta é composta por um long que identifica o utilizador e um inteiro
que traduz o número de posts publicados por aquele mesmo utilizador.
Estas duas informações vão constar de cada posição de um GArray entretanto criado
e que irá ser preenchido com a informação resultante de percorrer 
toda a GHashTable \texttt{users}.
De seguida ordenamos o referido GArray, o qual passará a conter todos os users,
bem como o total de post que cada um publicou, por ordem decrescente do número de posts.
Por último, inserirmos os N utilizadores com
mais posts de sempre na lista que é devolvida por esta função.

\subsubsection*{Query 3}
\label{sec:query3}

\textbf{“Dado um intervalo de tempo arbitrário,
obter o número total de posts (identificando perguntas e respostas separadamente) neste período.”}

Uma vez que na nossa estrutura de dados \texttt{day} já temos uma variável que nos diz o número total de perguntas
e o número total de respostas efetuadas num determinado dia, para sabermos o
número total de cada uma delas durante um intervalo de tempo criámos duas novas
variáveis na nossa função da query 3: \textsf{'n\_questions'} e \textsf{'n\_answers'},
e incrementámo-las com o valor das perguntas e respostas efetuadas, respetivamente,
durante os dias passados como parâmetro.

No final de percorridos todos os dias, adicionamos as duas variáveis da nossa função a um LONG\_pair, e retornamo-lo.


\subsubsection*{Query 4}
\label{sec:query4}

\textbf{“Dado um intervalo de tempo arbitrário, retornar todas as perguntas contendo uma determinada tag.
O retorno da função deverá ser uma lista com os IDs das perguntas ordenadas em cronologia inversa."}

Para esta query, tirando partido da nossa estrutura de dados \texttt{day}, 
que tem um GPtrArray com
os apontadores das perguntas do dia em questão, começamos por percorrer cada pergunta desse array
da última data passada como parâmetro e comparamos as suas tags com a tag passada como argumento.
Deste modo, se a pergunta tiver a tag desejada inserimos o seu ID num array dinâmico.
Vamos percorrendo os dias da “Date end” para a “Date begin”.

No final de consultarmos todas as perguntas do intervalo de tempo temos o array dinâmico preenchido,
e copiamos os IDs para uma LONG\_list.
Esta LONG\_list já tem os IDs das perguntas ordenadas por cronologia inversa uma vez que
fomos preenchendo o array dinâmico do dia mais recente para o dia mais antigo.



\subsubsection*{Query 5}
\label{sec:query5}

\textbf{“Dado um ID de utilizador,  devolver a informação do
seu perfil (short bio) e os IDs dos seus 10 últimos posts (perguntas ou respostas),
ordenados por cronologia inversa.”}

Para responder a esta query fazemos uma procura por Id na GHashTable * \texttt{users} que devolve
a estrutura do utilizador na qual está presente a informação do
seu perfil, char * shortbio, e um vetor, GArray * last\_posts, com os seus últimos posts que são
ordenados por cronologia inversa e do qual, em seguida, se copiam os 10 primeiros
elementos.

\subsubsection*{Query 6}
\label{sec:query6}

\textbf{“Dado um intervalo de tempo arbitrário, devolver os IDs das N respostas
com mais votos, em ordem decrescente do número de votos; O número de votos deverá
ser obtido pela diferença entre Up Votes (UpMod) e Down Votes (DownMod)."}

Nesta query utilizamos um GPtrArray auxiliar. Ao percorrer os dias do intervalo
de tempo fornecido como parâmetro inserimos no array auxiliar os apontadores das
respostas que iam aparecendo.

Depois de percorrer todos os dias temos um array auxiliar com todas as respostas
efetuadas nesse intervalo de tempo. Consequentemente, ordenamos o array pelo número
de votos (no inicio do array está a resposta com mais votos e no fim a resposta 
com menos votos).

Para sabermos as N respostas com mais votos, consultamos o nosso array auxiliar
e retiramos dele o ID das primeiras N respostas que aparecem.

Assim, temos a LONG\_list pedida, por ordem decrescente do número de votos.


\subsubsection*{Query 7}
\label{sec:query7}

\textbf{“Dado um intervalo de tempo arbitrário, devolver as IDs das N perguntas
com mais respostas, em ordem decrescente do número de respostas."}

Para a resolução desta query também recorremos a um array auxiliar, porém
desta vez com apontadores para perguntas.

Utilizamos o mesmo raciocínio da Query 6, mas agora fomos percorrendo o intervalo
de tempo e adicionando ao nosso array auxiliar os apontadores para as perguntas.

Depois disso, tivemos que verificar quantas respostas de cada pergunta também aconteceram
no intervalo de tempo passado como parâmetro. 
Para isso, tivemos que voltar a percorrer todo o intervalo de
tempo mas desta vez analisando as respostas. A cada resposta que aparecia, verificamos
se a pergunta associada a essa resposta se encontra no nosso array auxiliar 
(utilizando uma função já existente no glib: g\_ptr\_array\_find\_with\_equal\_func). 
Em caso afirmativo, significa que essa resposta aconteceu no intervalo de tempo pretendido 
e pertence a uma pergunta que também aconteceu no mesmo intervalo de tempo. 
Assim, incrementamos a variável \textsf{'n\_answers\_date'}, presente na nossa
estrutura \texttt{questions}, que é somente utilizada para esta query. 
Esta variável inicialmente tem o valor 0.

No fim, temos o nosso array de questions preenchido, e cada apontador de questions tem a respetiva 
variável auxiliar \textsf{'n\_answers\_date'} com o número de respostas feitas
no intervalo de tempo desejado. Ordenamos então esse array segundo o critério de maior número de respostas. 

Retiramos os primeiros N elementos do array e devolvemos os determinados IDs
numa LONG\_list, que se encontra então ordenada decrescentemente segundo o número de respostas.


\subsubsection*{Query 8}
\label{sec:query8}

\textbf{"Dado uma palavra, devolver uma lista com os IDs de
N perguntas cujos títulos a contenham, ordenados por cronologia inversa"}

Nesta query, são copiadas todas as perguntas para uma lista, GList * l, que é iterada
e na qual, em todas as perguntas até ao fim da lista ou até encontrar o número total de
perguntas pedido, é verificado se o seu título contém a palavra dada e, em caso positivo,
copia-se o Id para a estrutura a ser devolvida.



\subsubsection*{Query 9}
\label{sec:query9}

\textbf{"Dados os IDs de dois utilizadores, devolver as últimas
N perguntas (cronologia inversa) em que participaram dois utilizadores específicos.
Note que os utilizadores podem ter participado via pergunta ou respostas"}

Nesta query, são copiadas todas as perguntas para uma lista, GList * l, que é iterada
e na qual, em todas as perguntas até ao fim da lista ou até encontrar o número total de
perguntas pedido, é verificado se o utilizador 1 criou ou respondeu a este post e, em
caso positivo, verifica-se se o utilizador 2 criou ou respondeu a este post e, se
também isto for verdade, copia-se o Id do post para a estrutura a ser devolvida.



\subsubsection*{Query 10}
\label{sec:query10}

\textbf{“Dado o ID de uma pergunta, obter a melhor resposta.
Para isso, deverá usar a função de média ponderada abaixo: (score da resposta x 0.45)
+ (reputacao do utilizador x 0.25) + (votos recebidos pela resposta x 0.2) +
(comentarios recebidos pela resposta x 0.1)"}

Antes de mais, começamos por verificar se o id da pergunta existe, pois caso não
seja o id de uma pergunta a função devolverá -1. \par
No caso de o id ser válido, averiguamos quantas respostas tem essa mesma pergunta,
através da função getNAnswers, e calculamos para cada resposta a sua média ponderada.
Para o calculo da média de cada resposta socorremo-nos de várias funções.
Em primeiro lugar, procuramos na GHashTable dos \texttt{users} o user que obtivemos
em cada posição do GPtrArray \texttt{answers}, para de seguida conseguirmos obtermos
a sua reputação.
Posteriormente, obtivemos o score e número de comentários de cada resposta e calculamos
a média ponderada com esses três fatores (ignoramos os votos já que o score é
igual aos votos).
Finalmente, verificamos qual a resposta com melhor média ponderada para depois
devolvermos o seu id.


\subsubsection*{Query 11}
\label{sec:query11}

\textbf{“Dado um intervalo arbitrário de tempo, devolver os identificadores das N tags
mais usadas pelos N utilizadores com melhor reputação. Em ordem decrescente do número
de vezes em que a tag foi usada."}

Para esta query utilizamos 1 array auxiliar para guardar apontadores de tags.

Tirando partido da GList * \texttt{usersList}, presente na nossa estrutura 
\texttt{TCD\_community}, 
que possuí os apontadores para users ordenados consoante a sua reputação (da maior
para a menor), criamos uma nova GList * mas desta vez com apenas os N primeiros 
elementos da GList * usersList. Deste modo, temos uma GList com os N users com 
melhor reputação.

Depois disso, percorremos todos os dias do intervalo
de tempo tendo em atenção as perguntas, mais precisamente as suas tags.
Se a pergunta tiver sido efetuada por um user com melhor reputação (ver se existe
o user que fez essa pergunta na nossa GList auxiliar de users), pegamos nas tags
dessa pergunta (ainda todas juntas numa só string) e separamo-las de modo a termos
todas as tags individuais.

Na nossa estrutura de \texttt{tags} temos uma variável \textsf{‘value’}, com o valor 0,
que é utilizada somente para esta query. Incrementamos essa variável para cada
uma das tags contidas na pergunta. Adicionamos o apontador para essa tag ao nosso
array auxiliar (tendo em atenção se já a tínhamos adicionado ou não ao array).

No final de percorridos todos os dias, todas as perguntas e respetivas tags
temos um array de apontadores de tags, com todas as tags que apareceram durante
esse intervalo de tempo e que foram feitas por algum dos N users com melhor reputação,
com o seu respetivo número de ocorrências (armazenado na variável \textsf{‘value’}).

Ordenamos esse array pelo número de ocorrências (do maior para o menor) e
colocamos numa LONG\_list os primeiros N identificadores das tags desse array.


\subsection{Estratégias para melhorar o desempenho}
\label{sec:desempenho}

Um dos objetivos deste projeto é o desempenho. Para o garantirmos começamos por
estudar a melhor forma de ler e processar os dados armazenados
nos ficheiros XML e, pesadas todas as vantagens e desvantagens, optamos por fazer o
\textit{parser} dos dados com recurso ao SAX \textit{(Simple API for XML)},
em detrimento do DOM \textit{(Document Object Model)}. \par
O DOM é um modelo que representa documentos XML numa estrutura
em forma de árvore, designada de árvore DOM. Apesar do mesmo ser mais simples de utilizar,
tornava o programa substancialmente mais lento, visto que consumia mais memória porque os ficheiros
XML a serem processados eram grandes, resultando na construção de uma árvore DOM com toda a informação
contida nos mesmos, a qual permanecia na memória enquanto estivesse a ser utilizada. Acresce que
as diversas funcionalidades que o DOM possui, tais como, navegar pelos nós, remover, editar e apagar
os nós da árvore DOM, acabam por gerar uma sobrecarga da memória sendo, concomitantemente,
desnecessárias para a realização do projeto. \par
Por seu turno, a SAX efectua o \textit{parser} de ficheiros em formato XML, definindo
funções que são executadas quando determinado evento ocorre - ``callbacks''. O consumo de memória
é reduzido, comparativamente com o DOM, uma vez que a memória utilizada corresponde somente às
informações que estão sendo processadas pelas ``callbacks''. De facto, corridos alguns testes com
a DOM verificamos que o carregamento das estruturas com o mesmo demorava praticamente o dobro do
tempo do que com a SAX, factor determinante para a adoção da mesma. \par

A segunda medida que utilizamos para melhorar o desempenho do programa foi a criação
da estrutura de dados que armazena o tempo. Tendo em conta que cinco das onze queries
consideravam um intervalo de tempo, utilizamos um GPtrArray em detrimento de uma
Hashtable, que não tem os dados ordenados, porque consideramos que seria a forma mais
eficiente de percorrer os intervalos de tempo. Com esta solução, basta-nos saber qual
o índice do array onde se inicia o intervalo de tempo e de quantos dias é esse mesmo
intervalo para percorrermos o GPtrArray, conseguindo manter um custo baixo e constante
das procuras dos intervalos de tempo. \par

A terceira medida adotada foi o método de resolução da query 2. Antes de nos decidirmos
pela nossa implementação, testamos outras possibilidade para termos a certeza de que
esta solução seria a que obteria melhor desempenho.\par
Em primeiro lugar, testamos a possibilidade de termos uma GList, duplamente ligada,
com a informação contida na estrutura totalPosts e que contivesse apenas N elementos.
A solução passaria por ordenar a lista sempre que ``visitássemos'' a GHashTable users.
Acontece que o tempo desta implementação era substancialmente superior, na ordem dos
1.1s, 6.2s, 17s e 150s quando pedidos os 100,
500, 1.000 e 5.000 utilizadores com mais posts de sempre, respetivamente. \par
Outra tentativa de resolver esta query foi criar um GArray também com os dados
da estrutura totalPosts, só que este GArray teria somente N posições. O GArray
seria ordenado sempre que ``visitássemos'' a GHashTable users e quando estivesse
cheio, os valores que obteríamos da GHashTable users seriam sempre colocados na
N-ésima posição e reordenado o GArray por ordem decrescente do número de posts,
isto até acabarmos de iterar sobre a GHashTable users. Todavia, esta solução
mostrou-se ainda mais ineficiente, alcançando valores na ordem dos 3.8s, 23.9s,
50s e 294s quando pedidos os 100, 500, 1.000 e 5.000 utilizadores com mais posts
de sempre, respetivamente. \par
Assim, optamos pela implementação da query 2 descrita na secção Queries, visto que
o seu tempo é de cerca de 0.195s, 0.196s, 0.197s, 0.196s, 0.199s e 0.196s
quando pedidos os 100, 500, 1.000 e 5.000, 10.000, 50.000 utilizadores com mais
posts de sempre, respetivamente.


\section{Conclusões}
\label{sec:conclusao}

Face ao problema apresentado e analisando criticamente a solução proposta concluímos
que cumprimos todas as tarefas, conseguindo atingir os objetivos definidos. No decurso
do projeto socorremo-nos dos conhecimentos adquiridos nas unidades curriculares de
Algoritmos e Complexidade, Programação Imperativa, bem como Arquitetura de Computadores
de forma a equacionarmos a melhor solução para o problema apresentado, conseguindo
obter resultados bastante satisfatórios face ao que nos foi pedido. \par
Todavia, entendemos que há alguns aspetos da nossa solução que, eventualmente,
poderiam ser melhorados. Com efeito, infelizmente, não tivemos tempo para testar,
em termos de desempenho, todas as soluções possíveis e que pudessem fazer diferença
a nível de tempo em todas as queries.\par
Na verdade, dedicamos especial atenção à construção das nossas estruturas de dados,
bem como à forma como iríamos ler e processar os dados dos ficheiros XML, pois
verificamos que estas duas ações iriam ter um peso determinante em termos de
eficiência no trabalho. \par
Acresce que verificamos que alocamos memória sempre que fazemos um ``getter'' de um
\textit{char *} quando não seria necessário criar um espaço de memória para um
atributo que nunca é alterado na nossa estrutura de dados e que está disponível
até ao fim da execução do programa, pelo que ocupa espaço de memória desnecessário
e diminui o desempenho do programa. \par
Em suma, não obstante as potenciais melhorias que poderiam ser feitas no
programa, os testes por nós realizados, nas nossas máquinas, atingiram
um tempo de execução que consideramos bastante aceitável.



\end{document}
